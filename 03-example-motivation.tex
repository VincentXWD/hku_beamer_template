%---------------------------------------------------------
%	PRESENTATION BODY SLIDES
%---------------------------------------------------------
\section{Motivation} % Note all sections and subsections are automatically placed in your table of contents

%------------------------------------------------
\begin{frame}
    \frametitle{Title}
    \begin{center}
        \textbf{To use this template, just edit and add slides!} \newline
    \end{center}

    The remainder of these slides serves as an example of the features you can use: footnotes, citations, columns, mini pages, bullets, links, code, maths, etc.

    \begin{center}
        {\Huge\calligra Enjoy!}
    \end{center}
\end{frame}

%------------------------------------------------
\begin{frame}
    \frametitle{Intra-frame Footnotes and Citations I}
    
    Citation in Beamer works slightly differently from conventional cites as Beamer rewrites its footnote and citation functions. 
    A common issue is the duplication of footnotes in a frame when using \texttt{footcite}. \newline

    This paper \footcite{payne2014art}, that paper \footcite{payne2014art}, and another paper \footcite{payne2014art}. \newline

    And this paper \footcite{payne2014art}, that paper \footcite{payne2014art}, and another paper \footcite{payne2014art} again. 
\end{frame}

%------------------------------------------------
\begin{frame}
    \frametitle{Inter-frame Footnotes and Citations I}

    Another issue with \texttt{footcite} is the unwanted continuation of the footnote index. \newline

    This paper \footcite{payne2014art}, that paper \footcite{payne2014art}, and another paper \footcite{payne2014art}. \newline

    And this paper \footcite{payne2014art}, that paper \footcite{payne2014art}, and another paper \footcite{payne2014art} again. \newline

\end{frame}

%------------------------------------------------
\begin{frame}
    \frametitle{Intra-frame Footnotes and Citations II}
    
    This template provides a workaround for these issues. 
    Let's use the customized command \texttt{firstcite} when citing a reference in a frame for the first time, and \texttt{secondcite} for the following citations. \newline

    This paper \firstcite{payne2014art}, that paper \firstcite{payne2014art}, and another paper \firstcite{payne2014art}. \newline

    And this paper \secondcite{payne2014art}, that paper \secondcite{payne2014art}, and another paper \secondcite{payne2014art} again. 
\end{frame}

%------------------------------------------------
\begin{frame}
    \frametitle{Inter-frame Footnotes and Citations II}

    This workaround works for the inter-frame scenario as well. \newline

    This paper \firstcite{payne2014art}, that paper \firstcite{payne2014art}, and another paper \firstcite{payne2014art}. \newline

    And this paper \secondcite{payne2014art}, that paper \secondcite{payne2014art}, and another paper \secondcite{payne2014art} again. 
\end{frame}

%------------------------------------------------
\begin{frame}
    \frametitle{Columns}
    \framesubtitle{And Graphics}

    Check this slide to see how columns made the formatting look nice.

    \begin{columns}[t] % The "c" option specifies centered vertical alignment while the "t" option is used for top vertical alignment
        \begin{column}{0.5\textwidth} % Right column width
            % To add an image
            \begin{figure}[h!]
                \centering
                % \caption{Hawkins et al, 2015}
                \includegraphics[angle=0, width=4.5cm]{ganyu.jpeg}
                \label{Figure 1}
            \end{figure}
        \end{column}
        \begin{column}{0.5\textwidth} % Left column width
            \begin{figure}[h!]
                \centering
                % \caption{Hawkins et al, 2015}
                \includegraphics[angle=0, width=4.5cm]{keqing.jpeg}
                \label{Figure 2}
            \end{figure}
        \end{column}
    \end{columns}
\end{frame}

%------------------------------------------------
\begin{frame}
    \frametitle{Bullets}
    You can use bullets too: \newline
    \begin{itemize}
        \item Like this one \newline
        \item \& this one
    \end{itemize}
\end{frame}

%------------------------------------------------
\begin{frame}
    \label{Test} % For the link button for the Appendix slide
    \frametitle{Sub-bullets and Links}

    \begin{itemize}
        \item You can also nest sub-bullets
              \begin{itemize}
                  \item Sub-bullet 1
                  \item Sub-bullet 2
                  \item Sub-bullet 3
                  \item Sub-bullet 4 \newline
              \end{itemize}
    \end{itemize}

    \textbf{Below is a button that links to a slide in the appendix}

    \begin{center}
        \hyperlink{Figure}{\beamergotobutton{Go to graphs}}
    \end{center}
\end{frame}
